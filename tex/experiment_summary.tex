\chapter{Experimental Results}
\section{Overview of Experiments}
This document details the results of the experimental run conducted on \today.
The following experiment groups and individual experiments were executed:
\begin{itemize}
  \item \textbf{Group: Mnemosyne test with GPT 4.1}
  \begin{itemize}
    \item Mnemosyne GPT 4.1 Test
    \item Mnemosyne GPT 4.1 mini Test
    \item Mnemosyne GPT 4.1 nano Test
  \end{itemize}
\end{itemize}
\section{Summary of Results}
\begin{longtable}{p{0.4\textwidth}rrr}
\toprule
\textbf{Experiment} & \textbf{Overall Score} & \textbf{Entity F1} & \textbf{Relationship F1} \\
\midrule
\endfirsthead
\toprule
\textbf{Experiment} & \textbf{Overall Score} & \textbf{Entity F1} & \textbf{Relationship F1} \\
\midrule
\endhead
Mnemosyne GPT 4.1 Test & 90.55\% & 57.45\% & 14.86\% \\
Mnemosyne GPT 4.1 mini Test & 77.33\% & 71.11\% & 11.49\% \\
Mnemosyne GPT 4.1 nano Test & 16.39\% & 44.94\% & 0.00\% \\
\bottomrule
\caption{Summary of key performance metrics for each successful experiment.}
\end{longtable}
\clearpage
\section{Mnemosyne test with GPT 4.1}
\subsection{Mnemosyne GPT 4.1 Test}
A baseline tool test using the flagship GPT 4.1 model. This experiment processes the 'NER Test 1' document to verify the complete data extraction and refinement pipeline, establishing a benchmark for the 4.1 series.

\begin{figure}[!ht]
  \centering
  \includegraphics[width=0.48\textwidth]{figures/appendix\_fig/GRP002EXP001\_input\_wordcloud.png}
  \includegraphics[width=0.48\textwidth]{figures/appendix\_fig/GRP002EXP001\_entity\_wordcloud.png}
  \includegraphics[width=0.48\textwidth]{figures/appendix\_fig/GRP002EXP001\_relationship\_wordcloud.png}
  \includegraphics[width=0.48\textwidth]{figures/appendix\_fig/GRP002EXP001\_ontology\_graph.png}
  \caption{Visualizations for Mnemosyne GPT 4.1 Test. Top-left: Input Text. Top-right: Extracted Entities. Bottom-left: Relationship Types. Bottom-right: Type Ontology.}
\end{figure}
\clearpage
\subsubsection{Hyperparameters}
\begin{tabular}{ll}
\toprule
\textbf{Parameter} & \textbf{Value} \\
\midrule
Llm Provider & OpenAI \\
Llm Model & gpt-4.1 \\
Temperature & 0.1 \\
Chunk Size & 7 \\
Chunk Overlap & 2 \\
Num Refinement Cycles & 2 \\
\bottomrule
\end{tabular}

\subsubsection{Results}
\begin{tabular}{ll}
\toprule
\textbf{Metric} & \textbf{Value} \\
\midrule
Status & Success \\
Duration Seconds & 361.461499 \\
Final Nodes & 125 \\
Final Relationships & 393 \\
Overall Score & 90.55\% \\
Entity F1 Score & 57.45\% \\
Relationship F1 Score & 14.86\% \\
\bottomrule
\end{tabular}
\subsection{Mnemosyne GPT 4.1 mini Test}
A tool test using the more agile GPT 4.1-mini model. This experiment processes the 'NER Test 1' document to evaluate the performance and accuracy of a smaller, faster model within the same family.

\begin{figure}[!ht]
  \centering
  \includegraphics[width=0.48\textwidth]{figures/appendix\_fig/GRP002EXP002\_input\_wordcloud.png}
  \includegraphics[width=0.48\textwidth]{figures/appendix\_fig/GRP002EXP002\_entity\_wordcloud.png}
  \includegraphics[width=0.48\textwidth]{figures/appendix\_fig/GRP002EXP002\_relationship\_wordcloud.png}
  \includegraphics[width=0.48\textwidth]{figures/appendix\_fig/GRP002EXP002\_ontology\_graph.png}
  \caption{Visualizations for Mnemosyne GPT 4.1 mini Test. Top-left: Input Text. Top-right: Extracted Entities. Bottom-left: Relationship Types. Bottom-right: Type Ontology.}
\end{figure}
\clearpage
\subsubsection{Hyperparameters}
\begin{tabular}{ll}
\toprule
\textbf{Parameter} & \textbf{Value} \\
\midrule
Llm Provider & OpenAI \\
Llm Model & gpt-4.1-mini \\
Temperature & 0.1 \\
Chunk Size & 7 \\
Chunk Overlap & 2 \\
Num Refinement Cycles & 2 \\
\bottomrule
\end{tabular}

\subsubsection{Results}
\begin{tabular}{ll}
\toprule
\textbf{Metric} & \textbf{Value} \\
\midrule
Status & Success \\
Duration Seconds & 420.490714 \\
Final Nodes & 94 \\
Final Relationships & 326 \\
Overall Score & 77.33\% \\
Entity F1 Score & 71.11\% \\
Relationship F1 Score & 11.49\% \\
\bottomrule
\end{tabular}
\subsection{Mnemosyne GPT 4.1 nano Test}
A tool test using the most lightweight GPT 4.1-nano model. This experiment processes the 'NER Test 1' document to assess the capabilities of the smallest model, focusing on speed and cost-efficiency.

\begin{figure}[!ht]
  \centering
  \includegraphics[width=0.48\textwidth]{figures/appendix\_fig/GRP002EXP003\_input\_wordcloud.png}
  \includegraphics[width=0.48\textwidth]{figures/appendix\_fig/GRP002EXP003\_entity\_wordcloud.png}
  \includegraphics[width=0.48\textwidth]{figures/appendix\_fig/GRP002EXP003\_relationship\_wordcloud.png}
  \includegraphics[width=0.48\textwidth]{figures/appendix\_fig/GRP002EXP003\_ontology\_graph.png}
  \caption{Visualizations for Mnemosyne GPT 4.1 nano Test. Top-left: Input Text. Top-right: Extracted Entities. Bottom-left: Relationship Types. Bottom-right: Type Ontology.}
\end{figure}
\clearpage
\subsubsection{Hyperparameters}
\begin{tabular}{ll}
\toprule
\textbf{Parameter} & \textbf{Value} \\
\midrule
Llm Provider & OpenAI \\
Llm Model & gpt-4.1-nano \\
Temperature & 0.1 \\
Chunk Size & 7 \\
Chunk Overlap & 2 \\
Num Refinement Cycles & 2 \\
\bottomrule
\end{tabular}

\subsubsection{Results}
\begin{tabular}{ll}
\toprule
\textbf{Metric} & \textbf{Value} \\
\midrule
Status & Success \\
Duration Seconds & 260.402711 \\
Final Nodes & 72 \\
Final Relationships & 329 \\
Overall Score & 16.39\% \\
Entity F1 Score & 44.94\% \\
Relationship F1 Score & 0.00\% \\
\bottomrule
\end{tabular}